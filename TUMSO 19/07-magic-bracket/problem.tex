\documentclass[11pt,a4paper]{article}

\newcommand{\tumsoTime}{13:00 น. - 16:00 น.}
\newcommand{\tumsoRound}{2}

\usepackage{../tumso}
\usepackage[margin=0.5in]{geometry}
\usepackage[many]{tcolorbox}
\usepackage{mathspec}
\usepackage{setspace}
\usepackage{multicol}

\tcbset{
    sharp corners,
    colback = white,
    before skip = 0.2cm,
    after skip = 0.5cm
}

\newenvironment{bracket}[1][]{
    \ttfamily\obeylines\obeyspaces\frenchspacing
}{
    \ifdefined\NoExamples\else%
    \end{tabular}
    \fi%
}

% \newenvironment{brackett}[1][]{
%     \ttfamily
% }{
%     \ifdefined\NoExamples\else%
%     \end{tabular}
%     \fi%
%     \ignorespacesafterend
% }

\newenvironment{brackett}[1]{\ttfamily}{
    \text{#} \ignorespaces
}

\begin{document}

\begin{problem}{ตามหาวงเล็บที่หายไป (Magic Bracket)}{standard input}{standard output}{1 seconds}{256 megabytes}{100}

ในค่ำคืนที่เหน็บหนาว peteza ได้เข้านอนอย่างปกติ แต่พอเขาตื่นขึ้นมาในตอนเช้ากลับพบว่าโลกที่เขาอยู่นั้นเปลี่ยนไป เขาได้พบกับบางสิ่งบางอย่างที่แปลกตา นั่นก็คือ วงเล็บปริศนาที่โผล่ขึ้นมาบนท้องฟ้า

\newtcolorbox{peteza}{
    fontupper = \color{white},
    colback = black,
    boxrule = 0pt,
    align = center
}

\begin{center}
    \begin{bracket}
        ?()?[]??(?)???([]))??
    \end{bracket}
\end{center}

เขาสนใจมันเป็นอย่างมาก อยากรู้ว่ามันคืออะไร เขาจึงได้นั่งเขียนวงเล็บวนไปวนมา แล้วอยู่ๆเขาก็นึกสงสัยบางอย่างขึ้นว่า ถ้าให้ \begin{brackett}
    ?
\end{brackett}เป็นวงเล็บที่หายไป โดยสามารถแทนเป็นวงเล็บใดก็ได้จากวงเล็บ 4 แบบที่เหลือ \begin{brackett}
    \text{(}, \text{)}, \text{[}, \text{]}
\end{brackett}\hspace{-0.55em}จะสามารถวงเล็บที่ถูกต้อง โดยวงเล็บที่ถูกต้องมี 4 รูปแบบดังนี้
\begin{enumerate}
    \item \begin{brackett}
        \text{()}
    \end{brackett}
    \item \begin{brackett}
        \text{[]}
    \end{brackett}
    \item ประกอบขึ้นจากวงเล็บที่ถูกต้องตามรูปแบบที่ 1 หรือ 2 มาต่อกัน
    \item อยู่ในรูปแบบของ \begin{brackett}
        \text{(*)}
    \end{brackett}\hspace{-0.55em}หรือ \begin{brackett}
        \text{[*]}
    \end{brackett}\hspace{-0.55em}โดยที่ \begin{brackett}
        *
    \end{brackett}\hspace{-0.55em}เป็นวงเล็บที่ถูกต้องตามรูปแบบที่ 1, 2 หรือ 3
\end{enumerate}
ตัวอย่างเช่น \begin{brackett}
    \text{([]())}, \text{([()()]())}, \text{[]()[]([()()]())}
\end{brackett}\hspace{-0.55em}เป็นต้น

\InputFile

บรรทัดแรก ระบุจำนวนเต็ม $N$ แทนความยาวของ String $(1 \leq N \leq 500)$

บรรทัดที่ $2$ ระบุ String ขนาด $N$ หลัก ประกอบด้วย \begin{brackett}
\text{(}, \text{)}, \text{[}, \text{]}, \text{?}
\end{brackett}\hspace{-0.55em}โดย \begin{brackett}
    \text{?}
\end{brackett}\hspace{-0.55em}จะต้องเป็นวงเล็บแบบใดแบบหนึ่งจาก $4$ แบบต่อไปนี้ \begin{brackett}
    \text{(}, \text{)}, \text{[}, \text{]}
\end{brackett}\hspace{-0.55em}

\OutputFile
ตอบจำนวนเต็มเพียงหนึ่งตัว แทน เศษจากการหารด้วย $10^9 + 7$ ของวิธีการสร้างวงเล็บที่ถูกต้องทั้งหมด

\Scoring
ชุดทดสอบจะถูกแบ่งเป็น 5 ชุด จะได้คะแนนในแต่ละชุดก็ต่อเมื่อโปรแกรมให้ผลลัพธ์ถูกต้องในชุดทดสอบย่อยทั้งหมด
\begin{description}

\item[ชุดที่ 1 (12 คะแนน)] ประกอบด้วย \begin{brackett}
    \text{(, )}
\end{brackett}\hspace{-0.55em} เท่านั้น
\item[ชุดที่ 2 (12 คะแนน)] ประกอบด้วย \begin{brackett}
    \text{[, ]}
\end{brackett}\hspace{-0.55em} เท่านั้น
\item[ชุดที่ 3 (16 คะแนน)] ไม่ประกอบด้วย \begin{brackett}
    \text{?}
\end{brackett}
\item[ชุดที่ 4 (24 คะแนน)] จะมี $ 1 \leq N \leq 50$
\item[ชุดที่ 5 (36 คะแนน)] ไม่มีเงื่อนไขเพิ่มเติม 

\end{description}

\Examples

\begin{example}
\exmp{4
?()?
}{2
}%
\exmp{4
(??)
}{3
}%
\end{example}

\end{problem}

\end{document}
