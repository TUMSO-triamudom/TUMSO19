\documentclass[11pt,a4paper]{article}

\newcommand{\tumsoTime}{09:00 น. - 12:00 น.}
\newcommand{\tumsoRound}{1}

\usepackage{../tumso}
\usepackage[margin=0.5in]{geometry}
\usepackage[many]{tcolorbox}
\usepackage{mathspec}
\usepackage{setspace}
\usepackage{multicol}

\makeatletter
\newcommand{\tpmod}[1]{{\@displayfalse\pmod{#1}}}
\makeatother

\begin{document}

\begin{problem}{มอดทำไม (No More Mod)}{standard input}{standard output}{1 seconds}{256 megabytes}{100}

Jomnoiz ผู้ที่สนใจในคณิตศาสตร์เป็นพิเศษ ในวันหนึ่งเขาได้สงสัยว่า ถ้าเรามีจำนวนเต็ม $N$ และ $M$ \\มันจะมีกี่คู่อันดับ $(x, y)$ ที่ $1\leq x<y\leq N$ และทำให้ \[(M\tpmod{x})\tpmod{y}=(M\tpmod{y})\tpmod{x}\]

แต่เนื่องจากเขาติดแข่ง Hackathon ทำให้เขาไม่มีเวลาว่างพอเขาจึงต้องขอให้คุณช่วยคิด

\InputFile

บรรทัดแรก ระบุจำนวนเต็ม $Q$ แทนจำนวนคำถาม $(1 \leq Q \leq 10^3)$

บรรทัดที่ $2$ ถึง $Q + 1$ รับจำนวนเต็ม $N_i$ และ $M_i$ $(2 \leq N_i \leq 10^5, 1\leq M_i\leq 5\cdot 10^5)$

\OutputFile
มีจำนวน $Q$ บรรทัด ซึ่งบรรทัดที่ $i$ แสดงถึงคำตอบของคำถามที่ $i$

\Scoring
ชุดทดสอบจะถูกแบ่งเป็น 3 ชุด จะได้คะแนนในแต่ละชุดก็ต่อเมื่อโปรแกรมให้ผลลัพธ์ถูกต้องในชุดทดสอบย่อยทั้งหมด
\begin{description}

\item[ชุดที่ 1 (10 คะแนน)] จะมี $1\leq Q\leq 10, 2\leq N\leq 10^3, 1\leq M\leq 10^5$
\item[ชุดที่ 2 (40 คะแนน)] จะมี $1\leq Q\leq 100, 2\leq N\leq 10^4, 1\leq M\leq10^5$
\item[ชุดที่ 3 (50 คะแนน)] ไม่มีเงื่อนไขเพิ่มเติม 

\end{description}

\Examples

\begin{example}
\exmp{1
3 5
}{2
}%
\exmp{1
3 6
}{3
}%
\end{example}

\end{problem}

\end{document}
