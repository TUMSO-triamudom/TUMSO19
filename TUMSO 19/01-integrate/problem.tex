\documentclass[11pt,a4paper]{article}

\newcommand{\tumsoTime}{09:00 น. - 12:00 น.}
\newcommand{\tumsoRound}{1}

\usepackage{../tumso}

\begin{document}

\begin{problem}{อินทิเกรต}{standard input}{standard output}{0.25 seconds}{256 megabytes}{100}

แคลคูลัส เป็นสาขาหลักของคณิตศาสตร์ซึ่งพัฒนามาจากพีชคณิต เรขาคณิต และปัญหาทางฟิสิกส์ แคลคูลัสมีต้นกำเนิดจากสองแนวคิดหลัก คือ แคลคูลัสเชิงอนุพันธ์ (Differential Calculus) และ แคลคูลัสเชิงปริพันธ์ (Integral Calculus)

ในวันหนึ่งเด็กชาย Spad1e ได้เดินผ่านร้านหนังสือที่ห้างแห่งหนึ่ง ด้วยความที่เขาเป็นคนชอบอ่านหนังสือ เขาจึงได้เดินเข้าไปหาหนังสือที่จะมาอ่านในเวลาว่าง เขาได้พบกับหนังสือแคลคูลัส เขาได้นั่งอ่านทำความเข้าใจอยู่สักพักหนึ่งจนได้พบเข้ากับโจทย์ข้อหนึ่ง ซึ่งเขาต้องการจะหาคำตอบแต่เขากลับคิดมันไม่ออก (เพียงข้อเดียวในหนังสือเล่มนั้น) เขาเลยอยากให้คุณช่วย โดยโจทย์มีอยู่ว่า

ให้ $x$ เป็นจำนวนเต็มบวก จงหาค่าของ \[\int_0^x 3x^2\, \mathrm{d}x\]

โดยเขาก็ได้ให้ตัวอย่างเนื้อหาที่เขาอ่านก่อนหน้า และ คำใบ้ที่เขียนไว้ในหนังสือมาเผื่อคุณจะได้ใช้มัน
\begin{itemize}
    \item $\int \mathrm{d}x=x+C$
    \item $\int \cos x\mathrm{d}x=\sin x+C$
    \item $\int \sin x\mathrm{d}x=-\cos x+C$
    \item $\int e^x\mathrm{d}x=e^x+C$
\end{itemize}

โดยคำใบ้มีอยู่ว่า \fbox{$\mathrm{Riemann}$ $\mathrm{Sums}$} เผื่อคุณส่งสัยว่า $\mathrm{Riemann}$ $\mathrm{Sums}$ คืออะไร มันก็คือ \[\int_a^b f(x)dx=\lim_{x\to\infty}\sum_{i=1}^n f(x_i)\Delta x\] โดยที่ $\Delta x=\frac{b-a}{n}$ และ $x_i=a+i\Delta x$

หมายเหตุ : คำตอบอาจมีขนาดใหญ่ให้ตอบเป็นเศษที่เกิดจากการหารคำตอบด้วย $10^9 + 7$

\InputFile
ข้อมูลนำเข้ามีทั้งหมด $1$ บรรทัด
ประกอบด้วยจำนวนเต็ม $x$ $(1 \leq x \leq 10^{18})$

\OutputFile
ตอบจำนวนเต็มเพียงหนึ่งตัว แทนเศษจากการหารคำตอบด้วย $10^9+7$

\Scoring
ชุดทดสอบจะถูกแบ่งเป็น 2 ชุด จะได้คะแนนในแต่ละชุดก็ต่อเมื่อโปรแกรมให้ผลลัพธ์ถูกต้องในชุดทดสอบย่อยทั้งหมด

\begin{description}

\item[ชุดที่ 1 (31 คะแนน)] จะมี $ 1 \leq x \leq 10^9$

\item[ชุดที่ 2 (69 คะแนน)] ไม่มีเงื่อนไขเพิ่มเติม 

\end{description}

\Examples

\begin{example}
\exmp{0
}{0
}%
\exmp{1
}{1
}%
\end{example}

\end{problem}

\end{document}
