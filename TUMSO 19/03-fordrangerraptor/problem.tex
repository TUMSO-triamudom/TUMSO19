\documentclass[11pt,a4paper]{article}
 
\newcommand{\tumsoTime}{09:00 น. - 12:00 น.}
\newcommand{\tumsoRound}{1}
 
\usepackage{../tumso}
 
\begin{document}
 
\begin{problem}{ดุดัน ไม่เกรงใจใคร}{}{}{0.25 second}{32 megabytes}{100}
 
เรามาแก้โจทย์ตำโซ่ข้อนี้กันใน

สาม

สอง

หนึ่ง

ดุดัน ไม่เกรงใจใคร ฟอร์ด เรนเจอร์ แรปเตอร์ เจเนอเรชันใหม่ พบกันที่โชว์รูมฟอร์ดทั่วประเทศ

ความ \textbf{ดุดัน ไม่เกรงใจใคร} นั้นได้แผ่กระจายไปทั่วประเทศ เหล่านักเรียนที่มารวมตัวกันเพื่อแข่งตำโซ่ในวันนี้ ก็มีความปรารถนาที่จะแสดงความ \textbf{ดุดัน ไม่เกรงใจใคร} ใน\textit{แบบของคุณ}

เพื่อแสดงความ \textbf{ดุดัน ไม่เกรงใจใคร} ใน\textit{แบบของโปรแกรมเมอร์} เราจะทำการพิมพ์คำว่า FORD ตัวใหญ่ ๆ ในแบบต่าง ๆ

เช่น $N = 3$ จะมีหน้าตา \textit{แบบนี้}

\begin{verbatim}
### ### ### ##
#   # # # # # #
### # # ### # #
#   # # ##  # #
#   ### # # ##
\end{verbatim}

$N = 4$ ก็จะมีหน้าตา \textit{แบบนี้}
\begin{verbatim}
#### #### #### ###
#    #  # #  # #  #
#    #  # #  # #  #
#### #  # #### #  #
#    #  # ##   #  #
#    #  # # #  #  #
#    #### #  # ###
\end{verbatim}

\textit{This problem is brought to you by TUMSO 19th}

\YourWork

จงแสดง \textbf{ความดุดัน ไม่เกรงใจใคร} ใน\textit{แบบของคุณ}

\pagebreak

\InputFile
 
บรรทัดแรก จำนวนเต็ม $N$ $(3 \leq N \leq 1\,000)$ 
 
\OutputFile

ตัวอักษร FORD ขนาด $N$ ใน\textit{แบบของคุณ}ที่ \textbf{ดุดัน ไม่เกรงใจใคร}
 
\Scoring
ชุดทดสอบจะถูกแบ่งเป็น 4 ชุด จะได้คะแนนในแต่ละชุดก็ต่อเมื่อโปรแกรมให้ผลลัพธ์ถูกต้องในชุดทดสอบย่อยทั้งหมด
 
\begin{description}
\item[ชุดที่ 1 (7 คะแนน)] $N = 3$
\item[ชุดที่ 2 (5 คะแนน)] $N = 4$
\item[ชุดที่ 3 (19 คะแนน)] $N \leq 20$
\item[ชุดที่ 4 (69 คะแนน)] ไม่มีเงื่อนไขเพิ่มเติม
\end{description}
 
\end{problem}
 
\end{document}