\documentclass[11pt,a4paper]{article}

\newcommand{\tumsoTime}{13:00 น. - 16:00 น.}
\newcommand{\tumsoRound}{2}

\usepackage{../tumso}

\begin{document}

\begin{problem}{ครัวซองต์สลัด}{standard input}{standard output}{1 seconds}{256 megabytes}{100}

ด้วยความที่จุฬาได้เปิดหลักสูตรใหม่ Computer Engineering & Digital Technology (CEDT) หรือเรียกอีกอย่างว่าครัวซองต์ จึงได้มีการสร้างพื้นที่สำหรับเรียนหลักสูตรนี้โดยเฉพาะ จุดเด่นของหลักสูตรนี้คือจะมีการให้นักศึกษาได้ฝึกงานตั้งแต่ศึกษาอยู่ชั้นปีที่ 1

แพะ ผู้ที่สอนอยู่คณะวิศวกรรมศาสตร์ สาขาวิศวกรรมคอมพิวเตอร์มานาน จึงได้มีความคิดที่จะสร้างจุดศุนย์กลางของเทคโนโลยี โดยรวมบริษัทต่าง ๆ เพื่อที่จะง่ายต่อการฝึกงานของนักศึกษา แพะจึงได้ไปจ้างบริษัทรับเหมาก่อสร้างที่มีมะนาวเป็นประธานบริษัทอยู่

มะนาว ได้สั่งให้ทำการสร้างอาคารทั้งหมด $N$ อาคาร แต่ละอาคารจะมีหมายเลขของอาคารตั้งแต่ $1$ ถึง $N$ และได้สร้างทางเชื่อมระหว่างอาคารทั้งหมด $M$ เส้นทาง โดยความแปลกของทางเชื่อมเหล่านี้คือจะเป็นทางที่สามารถเดินไปได้ทางเดียว และบางทางเชื่อมอาจจะเป็นทางที่เดินเข้าไปยังอาคารเดิมจากที่มาก็ได้ (ตัวอย่างเช่นทางเชื่อมจากอาคารที่ $u$ ไปยังอาคารที่ $u$)

ttamx ประธานนักศึกษาในขณะนั้นเห็นว่าทางเชื่อมบางทางนั้นไม่จำเป็นจึงอยากทุบทิ้ง และยังมีบางทางเชื่อมที่ควรจะสร้างแต่ไม่ได้ถูกสร้างขึ้น โดยการดำเนินการตัดทางเชื่อมจะต้องใช้เงิน 1 บาท และ การสร้างทางเชื่อมเพิ่มก็จะต้องใช้เงิน 1 บาทเช่นกัน ttamx จึงสงสัยว่าเขาจะต้องใช้เงินน้อยสุดเท่าใดเพื่อให้ทุกอาคารมีทางเข้าและทางออกอาคารอย่างละ 1 แห่งเท่านั้น ด้วยความที่คุณเป็นเลขานุการของ ttamx คุณจึงต้องหาคำตอบนั้นเพื่อที่ ttamx จะได้ดำเนินการตามที่ต้องการ

หมายเหตุ : หากมีทางเชื่อมจากอาคารที่ $u$ ไปยังอาคารที่ $v$ นั่นคือ จะมีทางออกที่อาคารที่ $u$ 1 แห่ง และทางเข้าที่อาคารที่ $v$ 1 แห่ง

\InputFile

บรรทัดแรก ระบุจำนวนเต็ม $N, M$ $(1 \leq N \leq 300, 1 \leq M \leq Min(\frac{N(N-1)}{2}, 1000))$

บรรทัดที่ $2$ ถึง $M + 1$ ระบุจำนวนเต็ม $u_i$ และ $v_i$ แสดงถึงว่ามีทางเชื่อมจากอาคารที่ $u_i$ ไปยังอาคารที่ $v_i$ $(1 \leq u_i, v_i \leq N)$

\OutputFile
ตอบจำนวนเต็มเพียงหนึ่งตัว แทนจำนวนเงินที่น้อยที่สุดในหน่วยบาทที่ ttamx จะต้องใช้ในการดำเนินการตามที่ต้องการ

\Scoring
ชุดทดสอบจะถูกแบ่งเป็น 1 ชุด จะได้คะแนนในแต่ละชุดก็ต่อเมื่อโปรแกรมให้ผลลัพธ์ถูกต้องในชุดทดสอบย่อยทั้งหมด

\begin{description}

\item[ชุดที่ 1 (100 คะแนน)] ไม่มีเงื่อนไขเพิ่มเติม 

\end{description}

\Examples

\begin{example}
\exmp{3 1
2 2
}{2
}%
\exmp{3 2
2 2
2 2
}{3
}%
\end{example}

\end{problem}

\end{document}
